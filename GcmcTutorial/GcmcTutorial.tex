\documentclass[fleqn,a4paper,12pt]{article}
\usepackage{latexsym}
\usepackage{fullpage}
\usepackage{amsmath,amsfonts,amssymb}
\usepackage{graphicx}
\usepackage{subfigure}
\usepackage{float}
\usepackage{pifont}
\usepackage{authblk}
\usepackage{setspace}

\newcommand\vcent[1]{\ensuremath{\vcenter{\hbox{{#1}}}}}

\renewcommand{\thetable}{\Roman{table}}

\title{Tutorial: Using {\tt GromPy} in GCMC mode}

\author[1,2,*]{Ren\'e Pool}

\affil[1]{Centre for Integrative Bioinformatics Vrije Universiteit (IBIVU),
Vrije Universiteit Amsterdam, De Boelelaan 1081a, 1081HV Amsterdam, the Netherlands}
\affil[2]{Netherlands Bioinformatics Centre, Geert Grooteplein 28, 6525GA
Nijmegen, the Netherlands}
\affil[*]{Email: \texttt{r.pool@vu.nl}}

\date{\today}

\begin{document}
\maketitle

% ------------------------------------------------------------------------------
\begin{abstract}
This tutorial will let the user undergo all necessary steps to perform
simulations in the grand-canonical ensemble using the {\tt GROMACS} molecular
simulation library via the {\tt GromPy} interface. After going through this
tutorial you will be able to calculate an equation of state (EOS) of the
Lennard-Jones (LJ) fluid. The system consists of mono-atomic water molceules
defined in the MARTINI force field. The intermolecular interactions between
such molecules are modelled using the Lennard-Jones potential only. We assume
that you use {\tt bash} shell of the Linux operating system. If you are unfamiliar
with some Linux commands, you can find documentation in the corresponding {\tt
man} pages or on the web.
\end{abstract}

\section{Downloading and installing the required files}
\begin{itemize}
	\item Go to your favorite working directory: {\tt \$WORK}, e.g. {\tt 
	WORK=/home/user/Simulation}
	\begin{itemize}
		\item[$\to$] {\tt cd \$WORK}
    \end{itemize}
	\item Use {\tt git} to download the {\tt GromPy} files
	\begin{itemize}
		\item[$\to$] {\tt git clone git://github.com/GromPy/GromPy.git}
	\end{itemize}
	\item Go to the {\tt GromPy} directory
	\begin{itemize}
		\item[$\to$] {\tt cd GromPy}
	\end{itemize}
	\item Compile the {\tt GROMACS} source code
	\begin{itemize}
		\item[$\to$] {\tt cd gromacs}
		\item[$\to$] {\tt cd gromacs-4.0.7}
		\item[$\to$] {\tt tar -xzf gromacs-4.0.7-git.tar.gz} 
		\item[$\to$] {\tt cd gromacs-4.0.7-git}
		\item[$\to$] {\tt patch -p1 < ../grompy\_4.0.7\_patch.diff}
		\item[$\to$] {\tt ./bootstrap} 
		\item[$\to$] {\tt ./configure --enable-shared \char`\\ }\\
					   {\tt --enable-grompy \char`\\} \\
				       {\tt --prefix=\$PWD/install \char`\\}\\
				       {\tt CFLAGS=\char`\"-O2 -fPIC\char`\"}
		\item[$\to$] {\tt make -j 10}
		\item[$\to$] {\tt make install}
	\end{itemize}
	\item Save the {\tt GROMACS} install path to a variable
	\begin{itemize}
		\item[$\to$] {\tt GMXINSTALLDIR=\$PWD/install}
	\end{itemize}
	\item Go to the tutorial directory
	\begin{itemize}
	  	\item[$\to$] {\tt cd ../../../GcmcTutorial} 
	\end{itemize}
	\item We now need to ``{\tt source}'' the {\tt GROMACS} and {\tt GromPy}
	      environment variables\\
		  THIS NEEDS TO BE DONE WHENEVER YOU USE {\tt GromPy} IN A NEWLY OPENED
		  SHELL (SO REMEMBER THE {\tt GROMACS} INSTALLATION DIRECTORY)!!
	\begin{itemize}
	    \item[$\to$] {\tt source \$GMXINSTALLDIR/bin/GMXRC}
	  	\item[$\to$] {\tt source ./SourceGromPyEnv.sh \$GMXINSTALLDIR} 
  	\end{itemize}
\end{itemize}

\section{Calculation of the Lennard-Jones equation of state}
\begin{itemize}
  	\item The initial starting structure of the LJ system is an equilibrated one
  	containing $N=400$ LJ particles in a cubic box. Since it is equilibrated,
  	there is no need to energy minimize the system. We will however equilibrate it using
  	MD for the desired temperature $T=773~\rm K$, just to be sure. This is done
  	in two steps: (1) equilibration using a small time step and (2) production at
  	the target time step $dt=0.02~\rm ps$.
  	\begin{itemize}
  		\item[$\to$] {\tt cd GcmcLj}
  		\item[$\to$] {\tt mkdir mdeq}
  		\item[$\to$] {\tt cd mdeq}
  		\item[$\to$] {\tt grompp -f ../mdp/mdeq.mdp \char`\\}\\
  					 {\tt -c ../initstructure/water.gro \char`\\}\\
  					 {\tt -p ../top/W400.top}
  		\item[$\to$] {\tt mdrun}
  		\item[$\to$] {\tt cd ..}
  		\item[$\to$] {\tt mkdir mdprod}
  		\item[$\to$] {\tt cd mdprod}
  		\item[$\to$] {\tt grompp -f ../mdp/mdprod.mdp \char`\\}\\
  					 {\tt -c ../mdeq/confout.gro \char`\\}\\
  					 {\tt -p ../top/W400.top \char`\\}\\
  					 {\tt -maxwarn 1}
  	\end{itemize}
  	\item To enable sampling of LJ particle numbers in the range $N\in [0,450]$,
  	we need to generate $450$ {\tt .tpr} files (for $N=0$ we do not need to
  	perform energy calculations {\tt ;-)}). We delete LJ particles from the equilibrated
  	{\tt W400.gro} structure to generate the $N\in [1,399]$ range of {\tt .gro}
  	files and then we add LJ particles for the $N\in [401,450]$ range of {\tt
  	.gro} files. Remember that {\tt GromPy} only needs an equilibrated
  	{\em starting} structure (one that does not contain particle overlaps). So as
  	long as we start {\tt GromPy} using a $N\in [1,400]$ structure, it will run fine. For
  	the higher $N$ range we can just generate structures by adding a new LJ
  	particle the last particle on top of e.g. the last particle. Before
  	generating the {\tt .tpr} files, we also need to generate 450 topology
  	({\tt .top}) files.
  	\begin{itemize}
  	  	\item[$\to$] {\tt cd ../gro}
  	  	\item[$\to$] {\tt ln -sf ../mdprod/confout.gro ./W400.gro}
  	  	\item[$\to$] {\tt ./GenerateStartingStructures.sh 1 450 W400.gro}
  	  	\item[$\to$] {\tt cd ../top}
  	  	\item[$\to$] {\tt ./GenerateTopologies.sh 1 450 W400.top}
  	  	\item[$\to$] {\tt cd ../tpr}
  	  	\item[$\to$] {\tt ./GenerateTprs.sh 1 450}
  	\end{itemize}
  	\item We are now ready to perform grand-canonical simulations. In this
  	tutorial we will generate an equation of state in the $\mu VT$ ensemble at
  	$T=773~\rm K$
  	\begin{itemize}
  	  	\item[$\to$] {\tt cd ../}
  	  	\item[$\to$] {\tt ./RunEos.sh}
  	\end{itemize}
\end{itemize}

\section{Analysis}
\begin{itemize}
  	\item After performing the simulations, we will analyze the results by
  	generating block averages of the results. We postprocess the resulting file to obtain the equation of state: the
  	excess chemical potential as a function of density.
  	\begin{itemize}
  	  	\item[$\to$] {\tt ./DoBlockAveraging.sh}
  	\end{itemize}
  	\item Use your favorite plotting tool to check out the results in file
  	\newline {\tt \char`\"rho\_of\_mu\_with\_units.dat\char`\"}. If all works
  	well, the results should agree with the $NVT$ equation of state at $T=773~\rm K$.
\end{itemize}

\section*{Disclaimer}
This code is NOT error-proof. Please notify us if you run into any problems.

\end{document}